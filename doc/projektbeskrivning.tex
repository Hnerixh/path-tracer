\documentclass{article}
\usepackage{graphicx}
\usepackage[utf8]{inputenc}
\usepackage{amsmath}
\usepackage{amssymb}
\usepackage{mathtools}
\usepackage{tgtermes}
\usepackage{titling}
\usepackage[T1]{fontenc}
\usepackage{hyperref}
\usepackage{todonotes}
\usepackage[swedish]{babel}

\renewcommand{\contentsname}{Innehåll}

\newcommand{\subtitle}[1]{%
  \posttitle{%
    \par\end{center}
    \begin{center}\large#1\end{center}
    \vskip0.5em}%
}
\makeatletter
\def\@makechapterhead#1{%
  \vspace*{50\p@}%
  {\parindent \z@ \raggedright \normalfont
    \interlinepenalty\@M
    \Huge\bfseries  \thechapter.\quad #1\par\nobreak
    \vskip 40\p@
  }}
\makeatother



\begin{document}
\title{Projektbeskrivning}
\subtitle{Pathtracer}
\author{Henrik Henriksson <henhe459@student.liu.se> \\\\
Handledare: Daniel de Leng <daniel.de.leng@liu.se>}
\maketitle
\clearpage
\tableofcontents
\clearpage

\section{Introduktion}
Jag har skrivit en enkel pathtracer. En pathtracer är en typ av
Monte-Carlo baserad 3D renderare, som i princip simulerar ljusets
flöde. Metoden bygger på relativt enkla ideer och ger väldigt bra
reslutat, men är \emph{väldigt} beräkningsintensiv jämfört med andra
metoder.

Min implementation är inte kapabel till helt fotorealistiska bilder,
mest på grund av att stöd för texturer saknas. Den lyckas dock mäta
sig med Open-Source renderaren
Luxrender\footnote{\url{www.luxrender.net}} i vissa enklare
scener. Prestandan är dock värdelös, ungefär 3-4 gånger långsammare än
smallpt med liknande scenbeskrivningar.
\footnote{\url{http://www.kevinbeason.com/smallpt/}}.

\begin{figure}
  \begin{center}
  \includegraphics[width=8cm]{cornell_box1}
  \end{center}
  \caption{Min pathtracer}
\end{figure}

\begin{figure}
  \begin{center}
  \includegraphics[width=8cm]{cornell_ref}
  \end{center}
  \caption{Luxrender referens}
\end{figure}

\section{Bakgrundsinformation}
Pathtracing
\footnote{\url{http://en.wikipedia.org/wiki/Path_tracing}}
 är en relativt gammal metod för 3D-rendering, men har
trots det inte används särskillt mycket tidigare. Detta beror på hur
enormt beräkningsintensiv pathtracing är.

\subsection{Algoritm}
Algoritmen för pathtracing är relativt enkel.
\begin{verbatim}
för alla pixlar {
   1. Skicka ut en stråle från pixeln.
   2. Ta reda på vad strålen träffar först.
   3. Om träffen var en ljuskälla, spåra tillbaka till kameran.
      Annars: Skicka ut ny slumpmässig stråle från senaste träffen.
   4. GOTO 2.
   }
\end{verbatim}
Detta kallar jag för ett \emph{pass} i min kod. Enbart ett pass ger en
värdelös svart bild med några färgglada pixlar. Upprepar man
algoritmen några tusen gånger och tar medelvärdet kommer bilden
konvergera mot en ``fysiskt korrekt'' bild.

Självklart kan man inte låta algoritmen löpa amok, då kommer man rent
statistiskt sätt få några väldigt långa loopar. Därför avbryter man
spårningen med hjälp av antingen ett maxdjup eller rysk roulette,
alltså att man avbryter spårningen slumpmässigt med en viss
sannolikhet. Jag använder mig av en kombination av minimidjup och rysk
roulette.

\subsection{Material}
En stor del av 3D rendering är att få materialen rätt. Pathtracing
lämpar sig väl för fysisk simulering av material. Jag har
implementerat perfekt lambertiell reflektion, perfekt speculär
reflektion, refraktion med Schlicks approximation, samt ett
icke-fysikaliskt material som fungerar som påminner om en lackad matt
yta. Alla material skickar ut nya slumpmässiga strålar enligt en
distribution specifik för materialet, samt påverkar färg och
intensitet.

\subsection{Om pathtracing i allmänhet}
Medelvärdet av lyckade spårningar, \emph{samples}, kan räknas med
\emph{samples per pixel, spp}. Vissa scener kräver låg spp, till
exempel en scen med enbart perfekt reflektiva ytor kan kräva bara ett
pass. Andra scener, ofta små slutna scener med små ljuskällor, kan
kräva tiotusentals spp för att se bra ut.

Pathtracing gör också många saker som är svåra, eller rent av omöjliga
att implementera med rasterisering, helt ``gratis''. Till exempel
kaustik, mjuka skuggor, volymeriska ljuskällor, med mera, kommer
gratis som en följd av att man använder pathtracing.

Trots att pathtracing är beräkningsmässigt tungt finns en stor fördel
med metoden. Den är väldigt lätt att paralellisera. Tidigare nämnda
Luxrender paralelliserar över nätverk, CPU och GPU, medan spelmotorn Brigade
\footnote{\url{http://brigade3.com/}} lyckas köra pathtracing i
realtime på GPU. När antalet tillgängliga kärnor har ökat har även
användarna ökat. Tidigare var det ofta arkitekter och artister som
renderade stillbilder med pathtracing. De senare åren har dock
pathtracing används i flera stora filmproduktioner, till exempel Iron
Man 3, Pacific Rim och Gravity. \footnote{\url{https://www.solidangle.com/}}




\section{Milstolpar}
Dessa är mina ursprungliga milstolpar. Observera att de var skriva med
avseende på en \emph{raytracer}. Milstolpe 10 har ingenting att göra i
en pathtracer.

\begin{tabular}{|l|l|}
  \hline
\# & Beskrivning \\
\hline
1 & Få java att skriva en bild till disk. \\
2 & Planera på papper. \\
3 & Implementera grundläggande datatyper och klasser, till exempel vektorer. \\
4 & Gör det möjligt att beskriva scener. \\
5 & Skapa klasser för sfärer och plan som kan placeras i scener. \\
6 & Gör det möjligt att lägga till ljuskällor. \\
7 & Gör det möjligt att beskriva bakgrunden. \\
8 & Skapa en representation för ``kameran''. \\
9 & Implementera algoritmer för att detektera skärning med de hittils implementerade formerna. \\
10 & Rendera en enkel scen med enbart direkt ljus. \\
11 & Lägg till stöd för reflekterande ytor. \\
12 & Lägg till stöd för spekularitet och diffusion. \\
13 & Lägg till stöd för genomskinliga material. \\
14 & Implementera stöd för polygonmodeller. \\
15 & Lyckas med att importera något enkelt format för att beskriva polygoner. \\
16 & Lyckas rendera en enkel polygon modell. \\
17 & Lägg till möjligheten att använda en bild som bakgrund. \\
\hline
\end{tabular}

\section{Utveckling}
Under utvecklingsarbetet användes git, emacs, och IntelliJ IDEA. Till
en mindre grad användes även Eclipse, som emacsserver, samt Netbeans
för viss optimering.

\section{Implementationsbeskrivning}
\subsection{Milstolpar}
\paragraph{1. Skriva bild till disk.}
Helt genomförd. Använder inte Javas standardbibliotek.

\paragraph{2. Planera på papper}
Genomfört och arkiverat i \url{/dev/null}

\paragraph{3. Implementera grundläggande datatyper och klasser}
Helt genomfört, se \texttt{Vector3, Coordinate3, Ray, Normal}

\paragraph{4. Gör det möjligt att beskriva scener}
Helt genomfört. Se \texttt{doc/sc\_format\_specification.sc} och
\texttt{SceneParser}.

\paragraph{5. Skapa klasser för sfärer och plan}
Helt genomfört. Se \texttt{Shape, AbstractShape, Sphere, Plane, Triangle, Mesh}.

\paragraph{6. Ljuskällor}
Helt genomfört. Stöd för volymeriska ljuskällor. Stöd för punktljus
saknas, men var inte heller planerat. Se \texttt{LightMaterial}.

\paragraph{7. Bakgrund}
Helt genomfört. Kan beskrivas med antingen fast färg eller valfri bild
som kan läsas av Javas standardbibliotek. Se \texttt{Background}.

\paragraph{8. Skapa en representation för kameran}
Delvis genomförd. Kameran är låst till origo. Se \texttt{Camera,
  CameraWorker}.

\paragraph{9. Algoritmer för skärning}
Helt genomfört.

\paragraph{10. Rendera med enbart direkt ljus}
Inte implementerat, då denna punkt är delvis ogiltig i en pathtracer.

\paragraph{11. Reflekterande ytor}
Helt genomfört. Se \texttt{MirrorMaterial}.

\paragraph{12. Spekularitet och diffusion}
Helt genomfört. Finns enbart stöd för perfekt spekularitet och perfekt
diffusion. Viss begreppsförvirring här i de ursprungliga milstolparna.
Se \texttt{DiffuseMaterial, MirrorMaterial}.

\paragraph{13. Refraktion}
Delvis genomfört. Lambert-Beers lag är inte implementerad, så färgat
glas kommer inte bete sig fysiskt korrekt. Ser snyggt ut däremot.
Se \texttt{RefractiveMaterial}.

\paragraph{14. Polygonmodeller}
Helt genomfört. Stöd för någon slags acceleration saknas dock, så
skärningstest sker med $O(n)$. Se \texttt{MeshShape}.

\paragraph{15. Import av polygonmodell}
Helt genomfört. Import av \texttt{.raw} filer stöds. Enbart trianglar stöds.
Se \texttt{MeshShape}.

\paragraph{16. Rendera polygonmodell}
Helt genomfört, men \emph{låååångsamt} då rendering sker i $O(n)$. Med
någon slags acceleration kan detta fås till $O(log n)$, men detta är
långt utanför projektets tidsramar.

\paragraph{17. Bild som bakgrund}
Helt genomfört. Se 7.

\subsection{Dokumentation för programkod}

\begin{figure}
  \begin{centere}
    \includegraphics[width=15cm]{diagram}
  \end{centere}
\caption{\\ 1. Förhållandet mellan \texttt{Material} och \texttt{Shape} \\
  2. Vektorer, koordinater och strålar \\
  3. Hur progammet är ihopkedjat \\
  4. Hur output fungerar. }
\end{figure}
\subsubsection{Grov översikt}
En parser, \texttt{SceneParser}, läser av scenbeskrivningen och skapar
en \emph{kamera}, \texttt{Camera}, samt en \emph{scen},
\texttt{Scene}. Utifrån detta skapas en \emph{renderare},
\texttt{Renderer}. Renderaren är ansvarig för att styra kameran,
skriva till disk, samt vid behov avsluta renderingen när den är
klar. Kameran är ansvarig för att hålla ordning på renderingsresultat
hittils, samt köra renderingspass på begäran av renderaren. Kameran
frågar scenen vad en viss stråle borde ha för påverkan på
bilden. Scenen ansvarar för att ta emot en stråle från kameran och
pathtracea den. Scenen innehåller olika föremål, \texttt{Shape}, som
dessutom har material, \texttt{Material}.

\subsubsection{Hur programmet startar}
Programmets huvudsakliga ingångsfunktion är \texttt{main(String[]
  args)} i klassen \texttt{Main}. Det enda som görs här är att be
\texttt{SceneParser} att läsa scenfilen, för att sedan be renderaren
som fås tillbaka att rendera bilden.

\texttt{SceneParser} läser genom scenbeskrivningen och skapar en
\texttt{Renderer}. För nödvändiga värden som inte ges i
scenbeskrivningen används standardvärden. Först skapas en
\texttt{Scene} och en \texttt{Camera}, varefter renderaren skapas.

\subsubsection{Hur renderingen går till}
Renderaren renderar antingen upp till ett bestämt antal pass eller för
evigt. Renderaren kommer begära att kameran renderar några pass,
\texttt{writeInterval}, för att sedan skriva resultatet till
disk. Renderaren håller även koll på renderingstider.

\texttt{Camera} är en viktig klass, då den ansvarar för att hålla
ordning på bilden, samt distribuera ut arbete till flera
\texttt{CameraWorker}s. \texttt{Camera} delar upp varje pass i lika
många delar som det finns \texttt{CameraWorkers}. Dessa är oftast lika
många som antalet processorkärnor som Java kan se. \texttt{Camera}
kommer köa upp det begärda antalet pass, och sedan vänta tills alla
\texttt{CameraWorker}s är klara med sitt arbete. Arbetsuppgifter
skickas med \texttt{CameraJob} i en \texttt{BlockedQueue}, och
bekräftelse på att jobbet är utfört fås tillsammans med lite
information om lyckade och totala traces i en
\texttt{TraceResult}. \texttt{CameraWorkerInfo} används för att begära
att alla \texttt{CameraWorker}s ska avslutas.

Information om hur renderingen ser ut hittils hålls i en
endimensionell array av \texttt{Radiance}. \texttt{Radiance} används
både vid spårning och lagring. Vid lagring används medelvärdet av alla
\texttt{Radiance} som har spårats till pixeln. Vid spårning påverkas
färg och intensitet hos \texttt{Radiance} av de material som träffats.

\subsubsection{Hur skrivning fungerar}
\texttt{Renderer} anropar \texttt{Camera.render()}, som i sin tur ber
sin \texttt{Output} att skriva till disk. \texttt{Output} kommer i sin
tur använda en \texttt{Tonemapper} för att tonmappa varje pixel innan
bilden i helhet skrivs till disk i \texttt{.ppm} format.
\footnote{\url{http://netpbm.sourceforge.net/doc/ppm.html}}

\subsubsection{Hur spårningen fungerar}
En \texttt{CameraWorker} kommer fråga scenen efter \texttt{Radiance}
för varje pixel genom funktionen
\texttt{Scene.pathtrace()}. \texttt{pathtrace()} kommer iterera över
alla objekt i scenen för att hitta den närmaste
skärningspunkten. Detta sker i $O(n)$. Vid skärning beräknas en ny
stråle samt materialets påverkan på \texttt{Radiance}. Detta upprepas
tills man träffar en ljuskälla, eller slumpen avslutar spårningen.
Eftersom nya värden för \texttt{Radiance} beräknas med hjälp av
multiplikation, och multiplikation är kommutativt så spelar det ingen
roll om man räknar ut total \texttt{Radiance} för spårningen på väg
till ljuskällan eller ifrån ljuskällan. Därmed kan spårningen köras
helt iterativt, utan att man behöver lagra hela spårningen från kamera
till ljuskälla.

Ett \texttt{Material} ska tillhandahålla \texttt{getRandomRay()} och
\texttt{applyToRadiance()}. Den förstnämnda beräknar en ny slumpad
stråle från materialet med lämplig
distribution. \texttt{applyToRadiance()} förändrar den
\texttt{Radiance} som används vid spårningen enligt materialets
egenskaper. Ofta appliceras bara en färg.

En \texttt{Shape} ska tillhandahålla \texttt{intersection()} samt
\texttt{getMaterial()}. \texttt{intersection()} beräknar
skärningspunkten och returnerar en \texttt{IntersectionInfo} som
innehåller nödvändig info för fortsatt spårning.

\subsubsection{Allmänna klasser}
\paragraph{Coordinate3}
Används för att representera koordinater i rummet. Har få metoder, och
konverteras ofta till en \texttt{Vector3} vid beräkningar.

\paragraph{Vector3}
Används för nästan alla matematiska beräkningar. Innehåller diverse
matematiska funktioner. Skulle sett snyggare ut med operator overloading.

\paragraph{Normal}
Subklass till \texttt{Vector3} som används när man vill garantera att
en vektor är normaliserad.

\paragraph{Ray}
Stråle, alltså en \texttt{Coordinate3} och en
\texttt{Vector3}. Vektordelen av strålen normaliseras då \texttt{Ray}
initialiseras. \texttt{Ray} har också en tagg för att märka strålar
som färdas inuti någonting. Detta används vid refraktion.

\paragraph{RayMath}
Verktygsklass med några funktioner för generering av nya strålar samt
reflektans.

\paragraph{RandomGen}
Simpel slumptalsgenerator som används istället för den i standardbiblioteket.

\subsection{Användning av fritt material}
Kodmässigt har ingenting utanför Javas standardbibliotek använts. Vid
rendering har dock några modeller från Stanford använts.
\footnote{\url{http://www-graphics.stanford.edu/data/3Dscanrep/}} De
av Stanfords modeller jag har använt hämtades från
\url{http://graphics.cs.williams.edu/data/meshes.xml}, då dessa
modeller är i ett för mig enklare filformat samt till viss del
förbättrade.

3D programmet Blender har använts som verktyg för modellering och
konvertering samt till viss del referens. Blenders standardmodell
\emph{Suzanne} har använts.

\subsection{Designmönster som använts}
\subsubsection{Producer/Consumer}
Detta mönster används i \texttt{Camera} (producer) och
\texttt{CameraWorker} (consumer). \texttt{Camera} producerar
renderingsuppgifter (\texttt{CameraJob}) som \texttt{CameraWorker}
sedan konsumerar och renderar.

Fördelen med att använda detta mönster var att det gav ett relativt
enkelt sätt att implementera och kontrollera parallellismen på. Tyvärr
är det fortfarande ganska komplext jämfört med vissa funktioner i
andra språk. Användandet av \texttt{BlockingQueue} tillsammans med
Producer/Consumer resulterar också i att det blir svårt att utnyttja
all processorkraft till fullo.

\subsubsection{Observer}
Observer används i \texttt{CameraObserver, Output, Camera}. Genom att
använda Observer slipper man bry sig om att manuellt begära en
skrivning till disk i övrig kod. Så fort kameran är klar informeras
alla registrerade observatörer, så att de kan skriva till disk, eller
möjligtvis visa upp resultatet på skärmen i framtiden.

Utan Observer hade jag helt enkelt kallat på de korrekta funktionerna
manuellt när \texttt{Camera} blir klar.

\subsubsection{Strategy}
Strategy används i \texttt{Shape}, där man ger en \texttt{Material}
till konstruktorn.  Detta ger fördelen att användare av \texttt{Shape}
inte behöver bry sig om hur \texttt{Material} fungerar.

Utan detta mönster hade man skapat komplexitet i och med att
användaren av \texttt{Shape} i så fall behöver få tillgång till den
\texttt{Material} som hör till denna \texttt{Shape}.


\subsection{Användning av objektorientering}
\paragraph{1. Interface}
Interface används väldigt mycket i mitt projekt. Alla material och
shapes implementerar interfaces (\texttt{Material}, \texttt{Shape},
paketen \texttt{material} och \texttt{shape}). Detta gör att jag kan
behandla dessa väldigt generellt i stora delar av koden. Utan
objektorientering hade jag troligtvis löst detta med hjälp av en
touple av closures, då dessa bara implementerar rena funktioner, och
alltså egentligen inte behöver modifieras efter skapande.

\paragraph{2. Ärvning}
\texttt{Normal} ärver nästan all sin funktionalitet från
\texttt{Vector3}. \texttt{Normal} behöver därmed bara implementera
konstruktorer. Detta innebär att man har ett sätt att särskilja
normaliserade vektorer från vanliga vektorer, utan att förlora
kompabilitet och utan att skriva onödigt mycket kod. I ett
icke-objektorienterat språk skulle jag ha normaliserat manuellt,
alternativt skapat en separat typ för normaliserade vektorer. Detta
skulle inte ha varit något större problem då jag skulle ha använt
immutables och operator overloading.

\paragraph{3. Abstrakta klasser}
\texttt{AbstractShape} innehåller kod för att hålla ordning på
material. Trots att detta bara är några rader undviker man onödig
kodduplicering i fyra andra klasser. I ett icke objektorienterat språk
skulle jag ha löst detta med ett makro eller en template av något
slag.


\subsection{Designbeslut}
\paragraph{1. Skriva en egen slumptalsgenerator}
För att få vettig prestanda i en flertrådad pathtracer krävs att man
\emph{mycket} snabbt kan generera slumptal. Dessutom ville jag kunna
köra på Java 6, då jag hade tillgång till kraftfulla maskiner med
enbart Java 6. \texttt{Math.random()} ger slumpade tal mellan 0.0 och
1.0. Den fungerar bra entrådat, men flertrådat skapar den en flaskhals
av låsningar eftersom att den använder sig av \texttt{Random}. Alltså,
all slumptalsfunktionalitet i Java är trådsäker. Ett alternativ är att
skapa nya \texttt{Random} vid behov. Detta är dock slöseri med
resurser. Ett annat alternativ är att skicka med en \texttt{Random} i
funktionsanrop. Detta blir mycket onödig
spagettikod. \texttt{ThreadLocal} skulle kunna användas, men detta
introducerar nya klasser och mycket kod. Dessutom lär prestandan
minska.

Därför skrev jag min egen slumptalsgenerator, \texttt{RandomGen}. Den
använder ingen låsning, och är därmed snabb, men
ickedeterministisk. Eftersom att en ickedeterministisk
slumptalsgenerator fortfarande avger slumptal är det inte ett problem
för en pathtracer. I praktiken fungerar den mycket bra. Kvaliteten på
slumptalen är tillräckligt bra för min pathtracer. Denna lösning är
helt klart den enklaste och snabbaste, med enbart 14 rader kod, två
multiplikationer, en addition och en modulo per slumptal.

\paragraph{2. null}
På många platser i koden använder jag mig av null för att signalera
misslyckanden. Detta gör jag för att slippa skapa nya objekt i
relativt tajta loopar, till exempel i \texttt{for (Shape shape :
  shapes)} loopen i \texttt{Scene}, där \texttt{Shape.intersection()}
kan returnera null. Jag använder det även i till exempel material för
att signalera att \texttt{Material.getRandomRay()} inte ger upphov
till nya strålar. Här används det dock mest för att slippa lägga till
onödiga fält i \texttt{Ray}.

Istället för att använda null hade jag kunnat lägga till flaggor i
relevanta klasser. Detta hade dock blivit långsammare och fulare kod.

\paragraph{3. SceneParser}
Klassen \texttt{SceneParser} är den i särklass längsta klassen, och
även den klass som får flest varningar vid inspections. Den ser ut som
den gör för att jag inte vill dela upp den. Eftersom klassen bara
läser ur en textfil ville jag kod som trots att den blev lång skulle
vara enkel att läsa.

Om man läser koden med specifikationen uppe brevid så kommer man kunna
läsa hela koden uppifrån och ner utan problem. Det är någonting man
inte skulle kunna göra med vissa andra upplägg.

Det går säkert att korta ner koden, men jag tvivlar på att den går att
skriva på ett sätt som är enklare att förstå genom att dela upp den.

\paragraph{4. Returprodukt från consumer/producer}
För att få kontroll över hur mycket som renderas i taget, samt
``låsa'' renderingen innan skrivning så väntar \texttt{Camera} på att
alla \texttt{CameraWorker}s ska bli klara innan skrivning till
disk. Dessutom får man tillfälle att samla ihop information från
arbetarna.

Detta betyder att det kommer finnas tillfällen då alla
processorkärnor inte utnyttjas till max vid tillfällen runt
skrivning. Detta skulle kunna ändras till att alla
\texttt{CameraWorkers} släpps fria, och man låter dem rendera så
mycket de bara kan, utan någon kontroll, varefter man skriver till
disk från en annan tråd med vissa tidsintervall. Jag ville dock ha den
kontroll det gav att använda Consumer/Producer.

\paragraph{5. Implementera \texttt{Color} separat från \texttt{Radiance}}
På ytan ser \texttt{Color} och \texttt{Radiance} väldigt lika ut, men
eftersom \texttt{Color} och \texttt{Radiance} gör så olika saker
använder de sig inte av varandra. \texttt{Radiance} skulle till
exempel kunna ha varit implementerad med hjälp av \texttt{Color}, men
detta skulle ha lett till problem, då \texttt{Color} enbart har fält
som är \texttt{final}.

\paragraph{6. Användning av iteration i pathtracingfunktionen}
Om man tittar tillbaka på historiken i git ser man att jag använde mig
av en rekursiv \texttt{Scene.pathtrace()}. Jag ändrade den till en
iterativ version för att få koden mer ``samlad''. Den rekursiva
versionen krävde fler funktionsanrop till många olika ställen och
klasser. Dessutom blev det mycket enklare att implementera rysk
roulette i den iterativa versionen än i den rekursiva.

\paragraph{7. Skriva egen kod för att skriva bild till disk}
Jag har skrivit egen kod för att spara ner bilden till disk
(\texttt{Output, Tonemapper}. Jag tänkte använda
\texttt{BufferedImage} tills jag såg typdefinitionen
\texttt{setRGB(int x, int y,} \textbf{int} \texttt{rgb)}. Då uppfinner
jag hellre hjulet på nytt och får 16-bit output på köpet.

\paragraph{8. Skapa aldrig nya \texttt{Radiance}}
Ett prestandaproblem som upptäcktes med Netbeans profiler var att på
tok för många \texttt{Radiance} skapades, vilket ledde till mycket
jobb för garbage collektorn. Detta har nu ändrats så att mjukvaran
aldrig skapar fler \texttt{Radiance}, genom att varje
\texttt{CameraWorker} har en egen som den skickar vidare till
\texttt{Scene}, som i sin tur nollställer den, arbetar med den och
returnerar. Detta ledde till att koden blev en aning mer komplicerad,
men garbage collectorn gick från att ta 20\% av processortiden till
att ta >10\%. Även Vector3 skapas i överflöd, detta skulle kunna
avhjälpas med en Vector3Pool.


\section{Filer}
Dokumentation, samt de filer som krävs för att framställa
dokumentationen finns i \texttt{doc/}. Källkod finns i
\texttt{src/}. Renderingsresultat finns i \texttt{renders/}. Denna
mapp innehåller de flesta lyckade renderingar av större vikt som
gjorts. De flesta är dock renderade med hjälp av hårdkodade
scenklasser i Java. Om dessa klasser är av intresse finns de i
git-historiken. (ta kontakt så hittar jag exakta commits.) Några
scenbeskrivningar finns i \texttt{scenes/}. För testrenderingar
rekommenderas \texttt{cornell\_box.sc}, då denna visar upp mycket av
funktionaliteten utan att vara för långsam att
rendera. \texttt{dragon.sc} rekommenderas INTE att rendera, då en
sådan rendering kommer ta \emph{minst} en natt på sig innan den är
\emph{ igenkännbar }.

\section{Användarmanual}
\begin{verbatim}
$ java -jar PathTracer.jar inputScene.sc outputImage.ppm
\end{verbatim}
Och det är allt den kan göra. För att öppna filerna rekommenderas
varmt \texttt{feh -R 5 outputImage.ppm} eller Gimp. För att konvertera
till andra filformat kan imagequick användas:
\begin{verbatim}
$ convert outputImage.ppm outputImage.png
\end{verbatim}
Observera att varken feh eller Gimp klarar av att öppna 16-bit output
ppm. För att se resultatet måste det först konverteras med imagequick
enligt ovan.

IDEA klarar av att öppna \texttt{.ppm} filer i 8-bit format. Tyvärr
tror IDEA att det kan öppna 16-bit filer också, vilket leder till
mystiska resultat. (Någonslags konstig trunkering, förutom i extremt
ljusa eller mörka områden.)

\subsection{Scenbeskrivning}
Scener beskrivs enligt \texttt{doc/sc\_format\_specification.sc}. Man
bör observera följande: Kameran är fixerad i origo. Uppåt i bilden är
positiv $x$-led. Höger i bilden är positiv $y$-led. Innåt i bilden är
positiv $z$-led. Detta skiljer sig från flertalet 3D program, och kan
ställa till vissa problem. För att importera \texttt{.raw} filer bör
man först se till så att man bara har trianglar.

Det finns några exempel i \texttt{scenes/}. Dessa kan renderas som
beskrivet ovan. Dessutom finns exempel på renderingar i
\texttt{renders/}. Observera att vissa renderingar saknar motsvarande
scenbeskrivning i \texttt{scenes/}, då de har blivit renderade med
hårdkodade scenbeskrivningar.

\begin{figure}
  \begin{center}
  \includegraphics[width=8cm]{cornell_box2}
  \caption{Cornell box med textur på vägg och golv, samt en grön boll
    som är både matt och glansig.}
  \end{center}
\end{figure}

\begin{figure}
  \begin{center}
  \includegraphics[width=8cm]{bunny.png}
  \caption{Staford bunny. Tog på tok för lång tid att rendera.}
  \end{center}
\end{figure}


\begin{figure}
  \begin{center}
  \includegraphics[width=8cm]{first_cornell}
  \caption{Första vettiga bilden från min renderare.}
  \end{center}
\end{figure}


\section{Betygsambitioner}
Förhoppningen är att få en fyra eller femma.

\section{Utvärdering och erfarenheter}
Projektet har varit väldigt roligt att skriva. Samtidigt som Java är
helt fel språk att skriva en pathtracer i så är själva pathtracing
algoritmen väldigt rolig att implementera tillsammans med allt som
krävs runtomkring. Jag har lagt ner aningen för mycket tid på det här
projektet, men det hade jag förväntat mig.

Pathtracing är något som jag tycker att är mycket intressant, och jag
kommer nog fortsätta skriva på en pathtracer i sommar. Fast inte i
Java. Java är verkligen helt fel språk för en pathtracer. Inte för att
Java gör det jättedåligt, utan mer för att andra språk kan göra det så
mycket bättre. Jämför till exempel med smallpt, nämnt i
inledningen. Där implementeras paralellism med OpenMP och ett
kompilatorpragma. En rad, jämfört med mina fyra extra klasser i Java,
implementerar paralell pathtracing.

Något jag har haft problem med är att överlåta kompileringen till
IDEA. Samtidigt som jag gärna slipper ta hand om kompileringen själv
tycker jag att IDEA har bråkat på tok för mycket om småsaker. Vägrar
kompilera vissa filer, kompileringsfel som plötsligt försvinner utan
anledning, och så vidare. Av detta har jag lärt mig att man inte ska
lita på något IDE förrän de har bevisat att de duger.

\paragraph{Tips}
\begin{itemize}
\item Skriv en pathtraceri Java, det är roligt. Java är helt fel språk för
  det, och det är därför det blir roligt.
\item Få in designmönstren i koden så snabbt som möjligt.
\item Man kan skriva andra saker än spel.
\end{itemize}

\paragraph{Förbättringar till kursen}
\begin{itemize}
\item Se till så att alla dokument finns som
  \texttt{.pdf} också. Alla kan inte läsa \texttt{.odt}.
\item Gärna mer information på föreläsningar om hur man hanterar lite
  större projekt, versionshantering, tester, etc.
\end{itemize}


\end{document}
